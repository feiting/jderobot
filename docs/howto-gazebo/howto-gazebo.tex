\documentclass[oneside,a4paper,12pt]{article}
\usepackage[latin1]{inputenc}
\usepackage[spanish,activeacute]{babel}
\usepackage{geometry}
\usepackage{url}
\usepackage{graphicx}
\usepackage{a4wide}
\usepackage[normalsize]{subfigure}
\usepackage{titlesec}

\geometry{a4paper, left=3.5cm, right=2cm, top=3cm, bottom=2.0cm, headsep=1.5cm}

\begin{document}

\thispagestyle{empty}
\baselineskip 1.35\baselineskip
\vspace{2cm}
\begin{figure}[htb]
   \centerline{\resizebox{.50\textwidth}{!}{\includegraphics{figs/logo_urjc}}}
   \vspace{2cm}
   \centerline{\resizebox{.50\textwidth}{!}{\includegraphics{figs/logo}}}
\end{figure}

\begin{center}
  {\Large {\bf HOWTO DE GAZEBO }}
  \vspace{2cm}
\\V�ctor Manuel Hidalgo Bl�zquez\\Jos\'e Mar�a Ca�as Plaza 
\end{center}
\newpage


\section{Introducci\'on}

Gazebo es un simulador multirrobot para entornos abiertos, que es capaz de simular un conjunto de robots, sensores y objetos en un entorno tridimensional. Ademas incorpora simulacion f'isica realista, de modo que los objetos tienen masa, se pueden empujar unos a otros, etc. Entre los sensores que simula se incluyen c\'amaras como la sonyVID30, la canonVCC4 y c\'amaras genericas, asi como un par estereo.

El simulador Gazebo se encuadra dentro del entorno software abierto Player/Stage/Gazebo, y la p\'agina web oficial del proyecto es la  http://playerstage.sourceforge.net desde donde se puede descargar el c\'odigo fuente y ver la documentaci\'on actualizada.

Este manual describe como instalar \textit{gazebo}, para la posterior utilizaci�n de �ste en \textit{jde}. B'asicamente el simuldor gazebo se lanzar\'a como un programa separado, con el mundo a simular especificado en un fichero de configuraci'on. Ese mundo simulado tendr\'a uno o mas robots, dotados cada uno de ellos con sus sensores y sus actuadores. Con jdec podemos conectar un entorno jdec a cada uno de los robots de ese mundo simulado, para gobernar el comportamiento de ese robot. Para ello hay que cargar el driver de gazebo en jdec y configurarlo adecuadamente para corresponder los sensores y actuadores en jdec con los respectivos en gazebo.

% dependencias 
Antes de comenzar a instalar gazebo, debes tener aceleracion gr�fica y
ademas deben instalarse los siguientes paquetes:

\begin{itemize}
\item{\bf{SWIG}, Simplified Wrapper and Interface Generator }
\item{\bf{wxPython}, Python bindings for wxWidgets}
\item{\bf{ODE 0.5} Open Dynamics Engine}
\item{\bf{GDAL}, Geospatial Data Abstraction Library }
\end{itemize}
Incluso antes de instalar los paquetes que acabamos de comentar,
convienente configurar nuestro sistema con algunos paths y librerias
adicionales. Los path son los siguientes: 

\begin{verbatim}
export PATH=PATH:/usr/local/bin
export CPATH=CPATH:/usr/local/include
export LIBRARY_PATH=PATH:/usr/local/lib
export PKG_CONFIG_PATH=PKG_CONFIG_PATH/usr/local/lib/pkgconfig
export PYTHONPATH=PYTHONPATH/usr/local/lib/python2.3/site-packages
\end{verbatim}

\section{Instalacion en Ubuntu}

Las librerias que yo instale fueron las siguientes
\begin{verbatim} 
libjasper-1.701-1, libjasper-1.701-dev,libdc1394, libdc1394-dev,
gsl-bin, libgsl0, libgsl0-dev, libjpeg, libjpeg-dev, liftiff,
libtiff-dev, libpng, libpng-dev, libgnomecanvas2-common,
libgnomecanvas2-dev, imagemagick, python-dev, ffmpeg, libcv, libgeos-dev, libxerces-dev, libhdf4g-dev,  
libungif4-dev, libnetcdf-dev, netcdfg-dev, libcfitsio-dev, libpq-dev,
libgtk, automake1.9, autoconf,bison-1.35, python2.4-dev,libtdl. 
\end{verbatim}

\textbf{Importante}: seguramente haga falta alguna libreria m�s por
instalar, que en estos momentos no recuerdo. En ese caso enviar un
correo electronico diciendo la libreria en cuestion a alguna de las
siguientes direcciones: \\ 
\begin{center}
jmplaza@gsyc.escet.urjc.es,
hbmhidalgo@gmail.com.
\end{center}

Una vez habiendo instalado estas librerias y configurado nuestro
sistema con los path ya mencionados empezamos a instalar los paquetes
que har�n falta para instalar gazebo.\\

\subsection{Instalaci�n de SWIG}

Este paquete se puede instalar mediante un paquete debian, o atrav�s
de svn. Mi recomendaci�n es que sea a trav�s de svn. Si decidis
hacerlo con el paquete debian, podeis tener problemas de paquetes
rotos, en este caso para solucionarlo hacer lo siguiente: 
aptitude update, aptitude dist-upgrade. Si decidis hacerlo por svn,
ejecutar pa siguiente linea en consola: 
\begin{verbatim}
 svn co https://swig.svn.sourceforge.net/svnroot/swig/trunk swig
\end{verbatim} 

Los siguientes pasos son: 
\begin{verbatim}
 % cd swig
 % ./autogen.sh
 % ./configure 
 % make
 % make install
\end{verbatim}


\subsection{Instalaci�n de wxPython}

Hay que modificar el fichero source.list a mi me valio con tener los
mismos repositorios que los que hay en los ordenadores del laboratorio
de robotica: 


Los siguientes pasos son:

\begin{verbatim}
sudo apt-get update
sudo apt-get install python-wxgtk2.6 python-wxtools wx2.6-i18n
\end{verbatim}


\subsection{Instalaci�n de ODE}

Descargar ODE-0.5 de http://opende.sourceforge.net/snapshots/. Una vez
descargagado y descomprimido el archivo entrar al subdirectorio que se
desempaquetado y modificar el archivo de configuraci�n
config/user-settings  para habilitar la l�nea
OPCODE\_DIRECTORY=OPCODE. Una vez hecho esto abrir el archivo INSTALL y
ejecutar los make que aparacen. Despu�s copia los archivos en el
subdirectorio  /usr/local/lib a /usr/lib y los subdirectorios en 
/usr/local/include a  /usr/include.

Para probar que esta bien instalado ejecutar los ejemplos que se
encuentran en el subidrecotorio: ode/test

\subsection{Instalaci�n de GDAL}

Tienes que descargarte la version 1.2.1 de http://www.gdal.org/.
Despues haz los siguientes pasos: 
\begin{verbatim}
 tar xvzg gdal-1.2.1.tar.gz
 cd gdal-1.2.1
 ./configure --without-python
 make
 su
 make install
\end{verbatim}

\subsection{Instalaci�n de Gazebo desde su c\'odigo fuente}

La �ltima versi�n de gazebo la podemos descargar de la siguiente
pagina: http://playerstage.sourceforge.net/. Una vez descargado y
descomprimido llevar a cabo los siguientes pasos:
\begin{verbatim}
./configure
make
make install
\end{verbatim}


\section{Instalaci\'on en Fedora Core 5}


\section{Ejecuci�n de Gazebo}

Para poder ejecutar gazebo, ejecuta por linea de comandos lo
siguiente:

\begin{verbatim}
wxgazebo /usr/local/share/gazebo/worlds/xxxx.world
\end{verbatim}

Donde xxxx.world se corresponde con los mundos de ejemplo que hay en
dicho direcotrio. As� pues, para ejecuntar gazebo con jde, primero
tendremos que ejecutar el servidor de gazebo con wxgazebo, y despues
cargar los schemas correspondientes que usen gazebo. Para ello hay que
cargar primero el driver de gazebo.

\end{document}

